% Created 2018-08-21 Tue 08:42
% Intended LaTeX compiler: pdflatex
\documentclass[11pt]{article}
\usepackage[utf8]{inputenc}
\usepackage[T1]{fontenc}
\usepackage{graphicx}
\usepackage{grffile}
\usepackage{longtable}
\usepackage{wrapfig}
\usepackage{rotating}
\usepackage[normalem]{ulem}
\usepackage{amsmath}
\usepackage{textcomp}
\usepackage{amssymb}
\usepackage{capt-of}
\usepackage{hyperref}
\author{vinay}
\date{\today}
\title{}
\hypersetup{
 pdfauthor={vinay},
 pdftitle={},
 pdfkeywords={},
 pdfsubject={},
 pdfcreator={Emacs 26.1 (Org mode 9.1.3)}, 
 pdflang={English}}
\begin{document}

\tableofcontents

\section{Chapter one}
\label{sec:orgcd42728}

Go code is organined into packages, which are similar to libraries
or modules in other languages. 
A package consists of one or more \texttt{.go} source files in a single 
directory that define what the packages does.

Each source file begins with a \texttt{package} declaration, here \texttt{package main}
that states which package the file belongs to, followed by a list of
other packages that it imports, and then the declarations of the
program that are stored in that file.

the \texttt{fmt} package contains functions for printing formatted output and
scanning input.
\texttt{Println} is one of the basic output functions in \texttt{fmt};
it prints one or more values, seperated by spaces, with a newline
character at the end so that the values appear as a single line
of output.

Package \texttt{main} is special. It defines a standalone executable 
program, not a library.
Within package main the \emph{function} main is also special -- it's
where execution of the program begins.
Whatever \texttt{main} does is what the program does.
ofcourse, \texttt{main} will normally call upon functions in other packages
to do much of the work, such as function \texttt{fmt.Println}.


\subsection{Command Line Arguments}
\label{sec:orgd17e6f2}

The Variable \texttt{os.Args} is a \emph{slice} of strings.
Slices are a fundamental notion in Go.
A slice is a dynamically sized sequence \texttt{s} of array of elements
where individual elements can be accessed by \texttt{s[i]} and a 
contiguous subsequence as \texttt{s[m:n]}

The number of elements is given by \texttt{len(s)}.

The first element of \texttt{os.Args},
\texttt{os.Args[0]}, is the name of the command itself;
The other elements are arguments that were presented to the program
when it started execution


\begin{verbatim}
for _, arg := range os.Args[1: ] {
	s += sep + arg
	sep = " "
}
fmt.Println(s)
\end{verbatim}


here range produces two values, \texttt{index} and \texttt{value} of the element at that index
so arg handles \texttt{value} and \texttt{\_} handles the index


\subsection{Variables}
\label{sec:orgb8ae3e4}
The version of strings above uses short variable declaration
There are several other ways to declare a variable in go

\begin{verbatim}
s := ""
var s = ""
var s string
var s string = ""
\end{verbatim}
\end{document}
